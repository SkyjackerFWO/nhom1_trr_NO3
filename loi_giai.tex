\documentclass[11pt, oneside, a4paper]{article}
\usepackage[utf8]{vietnam}
\usepackage{amsmath, amssymb, amsfonts}
\usepackage{fancyhdr}
\usepackage{graphicx}
\usepackage{color}
\usepackage{hyperref}

%------------------- Canh lề trang -------------------
\voffset=-0.5in
\hoffset=0.2in
\setlength{\evensidemargin}{0.2in}
\setlength{\oddsidemargin}{0.2in}
\setlength{\topmargin}{0.01in}
\setlength{\headsep}{0.3in}
\setlength{\footskip}{0.6in}
\setlength{\textheight}{9.3in}
\setlength{\textwidth}{6.2in}

%------------------- Header và Footer -------------------
\pagestyle{fancy}
\fancyhf{}
\lhead{Đại học Phenikaa - Trường Công nghệ Thông tin}
\rhead{Toán Rời Rạc - Lời Giải}
\rfoot{\thepage}
\lfoot{Năm học 2024 - 2025}
\renewcommand{\headrulewidth}{0.4pt}
\renewcommand{\footrulewidth}{0.4pt}

%------------------- Định nghĩa môi trường -------------------
\newtheorem{bt}{Bài}[section]
\newtheorem{loigiai}{Lời giải}[bt]


%------------------- Bắt đầu tài liệu -------------------
\begin{document}

\begin{center}
    \Large \textbf{LỜI GIẢI BÀI TẬP MÔN TOÁN RỜI RẠC}\\
    \textbf{\underline{NĂM HỌC 2024 -- 2025}}\\
\end{center}

\tableofcontents
\newpage

%------------------- Phần 1: Kiến thức cơ sở -------------------
% \section{Phần 1: Kiến thức cơ sở (Logic)}

\begin{bt}
Chứng minh các biểu thức sau là hằng đúng bằng hai cách (lập bảng chân trị và dùng luật logic): 
\begin{itemize}
    \item[a)] $((P \rightarrow Q) \land P) \rightarrow Q$
    \item[b)] $P \land Q \rightarrow P$
    \item[c)] $\lnot (P \land Q) \land P \rightarrow \lnot Q$
    \item[d)] $(P \rightarrow Q \land R) \rightarrow ((P \rightarrow Q) \land (P \rightarrow R))$
    \item[e)] $((P \land Q) \leftrightarrow P) \rightarrow (P \rightarrow Q)$
\end{itemize}
\end{bt}

\begin{loigiai}
\begin{itemize}
    \item[a)] 
    \begin{itemize}
        \item \textbf{Lập bảng chân trị:}
        \begin{center}
            \begin{tabular}{|c|c|c|c|c|}
                \hline
                $P$ & $Q$ & $P \rightarrow Q$ & $((P \rightarrow Q) \land P)$ & $((P \rightarrow Q) \land P) \rightarrow Q$ \\
                \hline
                T & T & T & T & T \\
                T & F & F & F & T \\
                F & T & T & F & T \\
                F & F & T & F & T \\
                \hline
            \end{tabular}
        \end{center}
        Biểu thức luôn đúng.
        \item \textbf{Dùng luật logic:}
        \[
        ((P \rightarrow Q) \land P) \rightarrow Q \equiv (\lnot P \lor Q) \land P \rightarrow Q \equiv Q
        \]
    \end{itemize}

    \item[b)] Tương tự, ta có thể lập bảng chân trị hoặc áp dụng luật logic để chứng minh.
\end{itemize}
\end{loigiai}

\newpage


\end{document}