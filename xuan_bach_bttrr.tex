
\documentclass[12pt]{article}
\usepackage[utf8]{inputenc}
\usepackage[vietnamese]{babel}
\usepackage{amsmath, amssymb}
\usepackage{geometry}
\usepackage{enumitem}
\geometry{a4paper, margin=2.5cm}
\title{Lời giải Bài Tập Chương 1 - Logic cơ bản}
\author{}
\date{}

\begin{document}
\maketitle

\noindent\textbf{Bài 1.1:} \textit{Chứng minh các biểu thức sau là hằng đúng bằng hai cách (lập bảng chân trị và dùng luật logic).}


\textbf{a) } $((P \rightarrow Q) \land P) \rightarrow Q$

\textbf{– Bảng chân trị:}

\begin{center}
\begin{tabular}{|c|c|c|c|c|}
\hline
P & Q & $P \rightarrow Q$ & $(P \rightarrow Q) \land P$ & Biểu thức \\
\hline
T & T & T & T & T \\
T & F & F & F & T \\
F & T & T & F & T \\
F & F & T & F & T \\
\hline
\end{tabular}
\end{center}

\textbf{– Dùng luật logic:}

\[
((P \rightarrow Q) \land P) \rightarrow Q \equiv \text{áp dụng Modus Ponens} \Rightarrow Q
\]

---

\textbf{b) } $P \land Q \rightarrow P$

\textbf{– Bảng chân trị:}

\begin{center}
\begin{tabular}{|c|c|c|c|}
\hline
P & Q & $P \land Q$ & Biểu thức \\
\hline
T & T & T & T \\
T & F & F & T \\
F & T & F & T \\
F & F & F & T \\
\hline
\end{tabular}
\end{center}

\textbf{– Dùng luật logic:} Từ $P \land Q$ \rightarrow P$ theo luật loại bỏ hội.

---

\textbf{c) } $\neg(P \land Q) \land P \rightarrow \neg Q$

\textbf{– Bảng chân trị:}

\begin{center}
\begin{tabular}{|c|c|c|c|c|c|}
\hline
P & Q & $P \land Q$ & $\neg(P \land Q)$ & $\neg(P \land Q) \land P$ & Biểu thức \\
\hline
T & T & T & F & F & T \\
T & F & F & T & T & T \\
F & T & F & T & F & T \\
F & F & F & T & F & T \\
\hline
\end{tabular}
\end{center}

\textbf{– Dùng luật logic:} Nếu $P$ đúng và $P \land Q$ sai thì $Q$ phải sai $\Rightarrow \neg Q$ đúng.

---

\textbf{d) } $(P \rightarrow (Q \land R)) \rightarrow ((P \rightarrow Q) \land (P \rightarrow R))$

\textbf{– Bảng chân trị:}

\begin{center}
\begin{tabular}{|c|c|c|c|c|c|c|c|c|}
\hline
P & Q & R & $Q \land R$ & $P \rightarrow (Q \land R)$ & $P \rightarrow Q$ & $P \rightarrow R$ & RHS & Biểu thức \\
\hline
T & T & T & T & T & T & T & T & T \\
T & T & F & F & F & T & F & F & T \\
T & F & T & F & F & F & T & F & T \\
T & F & F & F & F & F & F & F & T \\
F & T & T & T & T & T & T & T & T \\
F & T & F & F & T & T & T & T & T \\
F & F & T & F & T & T & T & T & T \\
F & F & F & F & T & T & T & T & T \\
\hline
\end{tabular}
\end{center}

\textbf{– Dùng luật logic:} Nếu $P$ đúng thì $Q \land R$ đúng $\Rightarrow Q$ và $R$ đều đúng.\\
Nếu $P$ sai thì $P \rightarrow X$ luôn đúng.\\
\textbf{→ Biểu thức luôn đúng ⇒ hằng đúng.}

---

\textbf{đ) } $((P \land Q) \leftrightarrow P) \rightarrow (P \rightarrow Q)$


\vspace{1em}
\noindent\textbf{Bài 1.2:} \textit{Sử dụng quy tắc suy diễn trong mệnh đề logic.}

\begin{enumerate}[label=\alph*)]
\item $((X_1 \rightarrow X_2) \land (\neg X_3 \lor X_4) \land (X_1 \lor X_3)) \rightarrow (\neg X_2 \rightarrow X_4)$\\
Giả sử $\neg X_2$ đúng. Từ $X_1 \rightarrow X_2$ suy ra $X_1$ phải sai. Khi đó, $X_3$ phải đúng. Với $\neg X_3 \lor X_4$ và $X_3$ đúng $\Rightarrow X_4$ đúng.\\
$\Rightarrow$ mệnh đề đúng.

\item Lập luận: Nếu được thưởng thì An đi Đà Lạt. Nếu đi Đà Lạt thì An thăm Thiền Viện. An không thăm Thiền Viện $\Rightarrow$ không đi Đà Lạt $\Rightarrow$ không được thưởng.\\
\textbf{Suy luận đúng (Modus Tollens)}.
\end{enumerate}

\vspace{1em}
\noindent\textbf{Bài 1.3:} \textit{Dịch các câu thành biểu thức logic.}

Gọi:
\begin{itemize}
  \item $R(x)$: x là chim ruồi
  \item $S(x)$: x có màu sặc sỡ
  \item $L(x)$: x là chim lớn
  \item $H(x)$: x sống bằng mật ong
  \item $G(x)$: x có màu xám
  \item $N(x)$: x là chim nhỏ
\end{itemize}

\begin{enumerate}[label=\alph*)]
\item $\forall x (R(x) \rightarrow S(x))$
\item $\forall x (L(x) \rightarrow \neg H(x))$
\item $\forall x (\neg(L(x) \land H(x)) \rightarrow G(x))$
\item $\forall x (R(x) \rightarrow N(x))$
\end{enumerate}

\vspace{1em}
\noindent\textbf{Bài 1.4:} \textit{Dịch các câu thành biểu thức logic có lượng từ.}

Gọi $L(x, y)$: x yêu y.

\begin{enumerate}[label=\alph*)]
\item $\forall x\, L(x, Mai)$
\item $\forall x\, \exists y\, L(x, y)$
\item $\exists x\, \forall y\, L(y, x)$
\item $\neg \exists x\, \forall y\, L(x, y)$
\item $\exists x\, [\forall y\, \neg L(x, y) \lor \forall y\, \neg L(y, x)]$
\item $\exists x\, \neg L(Nam, x)$
\item $\exists x\, [\forall y\, L(y, x) \land \forall z\, (\forall y\, L(y, z) \rightarrow z = x)]$
\item $\exists x\, \exists y\, [x \ne y \land L(Tuấn, x) \land L(Tuấn, y) \land \forall z\, (L(Tuấn, z) \rightarrow (z = x \lor z = y))]$
\end{enumerate}

\vspace{1em}
\noindent\textbf{Bài 1.5:} \textit{Kiểm tra mô hình suy diễn.}
Cho hai mệnh đề:
\[
\text{(1) } \forall x (P(x) \rightarrow (Q(x) \land R(x))), \quad
\text{(2) } \forall x (P(x) \land F(x))
\]

Từ (2) suy ra \(P(x)\) đúng với mọi \(x\), kết hợp với (1) ta được:
\[
P(x) \rightarrow (Q(x) \land R(x)) \Rightarrow Q(x) \land R(x) \Rightarrow R(x)
\]
Mà (2) cũng cho \(F(x)\) đúng với mọi \(x\), nên \(R(x) \land F(x)\) đúng với mọi \(x\).

Do đó, ta suy ra:
\[
\boxed{\forall x (R(x) \land F(x))}
\]

Suy luận là \textbf{đúng}.


\vspace{1em}
\noindent\textbf{Bài 1.6:} \textit{Chứng minh các cặp mệnh đề sau tương đương hoặc không.}


\begin{enumerate}[label=\alph*)]

\item $(P \rightarrow Q) \rightarrow R$ và $P \rightarrow (Q \rightarrow R)$\\
\textbf{Không tương đương.}\\
Xét phản ví dụ: $P = \text{T}, Q = \text{F}, R = \text{T}$. Khi đó:
\[
(P \rightarrow Q) = \text{F}, \quad (P \rightarrow Q) \rightarrow R = \text{T}
\]
\[
(Q \rightarrow R) = \text{F} \rightarrow \text{T} = \text{T}, \quad P \rightarrow (Q \rightarrow R) = \text{T}
\]
Tuy cho kết quả giống nhau, nhưng với $P = T, Q = T, R = F$ thì:
\[
(P \rightarrow Q) = T, \quad (P \rightarrow Q) \rightarrow R = F
\]
\[
Q \rightarrow R = T \rightarrow F = F, \quad P \rightarrow (Q \rightarrow R) = T \rightarrow F = F
\]
Có những trường hợp kết quả khác nhau.\\
$\Rightarrow$ Hai mệnh đề \textbf{không tương đương}.

\item $\neg P \leftrightarrow Q$ và $P \leftrightarrow \neg Q$\\
\textbf{Tương đương.}\\
Vì phủ định hai vế và đổi vị trí vẫn giữ ý nghĩa tương đương logic.\\
Có thể xác nhận bằng bảng chân trị, hai mệnh đề luôn cho cùng giá trị.

\item $\neg (P \leftrightarrow Q)$ và $\neg P \leftrightarrow Q$\\
\textbf{Không tương đương.}\\
Xét bảng chân trị:

\begin{center}
\begin{tabular}{|c|c|c|c|c|c|}
\hline
P & Q & $P \leftrightarrow Q$ & $\neg (P \leftrightarrow Q)$ & $\neg P$ & $\neg P \leftrightarrow Q$ \\
\hline
T & T & T & F & F & F \\
T & F & F & T & F & T \\
F & T & F & T & T & T \\
F & F & T & F & T & F \\
\hline
\end{tabular}
\end{center}
Hai mệnh đề cho giá trị khác nhau trong một số trường hợp ⇒ không tương đương.

\item $\neg \exists x \forall y\, P(x, y)$ và $\forall x \exists y\, \neg P(x, y)$\\
\textbf{Tương đương.}\\
Theo quy tắc phủ định lượng từ:
\[
\neg \exists x \forall y\, P(x, y) \equiv \forall x \exists y\, \neg P(x, y)
\]

\item $(\forall x\, P(x)) \land A$ và $\forall x (P(x) \land A)$\\
\textbf{Tương đương nếu $A$ không chứa biến $x$.}\\
Vì trong trường hợp đó, $A$ là một mệnh đề độc lập có thể phân phối vào hoặc ra ngoài lượng từ mà không ảnh hưởng logic.

\item $(\exists x\, P(x)) \land A$ và $\exists x (P(x) \land A)$\\
\textbf{Tương đương nếu $A$ không chứa biến $x$.}\\
Giải thích tương tự câu (e): nếu $A$ không phụ thuộc vào $x$, việc đặt nó bên trong hoặc ngoài lượng từ $\exists$ không làm thay đổi ý nghĩa mệnh đề.


\vspace{1em}
\noindent\textbf{Bài 1.7:} \textit{Kiểm tra suy luận.}

\begin{enumerate}[label=\alph*)]
\item Từ chuỗi điều kiện, suy ra $X_5 = F \Rightarrow X_4 \lor X_5 = F \Rightarrow X_3 = F \Rightarrow (\neg X_1 \lor X_2) = F \Rightarrow X_1 = T, X_2 = F$\\
$\Rightarrow$ suy luận đúng.

\item Biểu thức $((P \rightarrow ((Q \lor R) \land S)) \land P) \rightarrow ((Q \lor R) \land S)$ là hằng đúng do áp dụng Modus Ponens.
\end{enumerate}

\vspace{1em}
\noindent\textbf{Bài 1.8:} \textit{Dịch sang biểu thức mệnh đề.}

\begin{itemize}
  \item $P(x)$: x là đứa bé
  \item $Q(x)$: x tư duy logic
  \item $R(x)$: x cai quản cá sấu
  \item $S(x)$: x bị coi thường
\end{itemize}

\begin{enumerate}[label=\alph*)]
\item $\forall x (P(x) \rightarrow \neg Q(x))$
\item $\forall x (R(x) \rightarrow \neg S(x))$
\item $\forall x (\neg Q(x) \rightarrow S(x))$
\item $\forall x (P(x) \rightarrow \neg R(x))$
\item Không suy ra được (d) từ (a), (b), (c) vì không có liên hệ trực tiếp giữa $P(x)$ và $R(x)$.
\end{enumerate}

\end{document}
